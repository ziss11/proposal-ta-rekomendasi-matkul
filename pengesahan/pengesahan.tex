\begin{flushleft}
  % Ubah kalimat berikut sesuai dengan nama departemen dan fakultas
  \textbf{Departemen Teknik Komputer - FTEIC}\\
  \textbf{Institut Teknologi Sepuluh Nopember}\\
\end{flushleft}

\begin{center}
  % Ubah detail mata kuliah berikut sesuai dengan yang ditentukan oleh departemen
  \underline{\textbf{EC224701 - PRA TUGAS AKHIR - 2 SKS}}
\end{center}

\begin{adjustwidth}{-0.2cm}{}
  \begin{tabular}{lcp{0.7\linewidth}}

    % Ubah kalimat-kalimat berikut sesuai dengan nama dan NRP mahasiswa
    Nama Mahasiswa     & : & Abdul Azis                                                   \\
    Nomor Pokok        & : & 0721 19 4000 7004                                            \\

    % Ubah kalimat berikut sesuai dengan semester pengajuan proposal
    Semester           & : & Ganjil 2022/2023                                             \\

    % Ubah kalimat-kalimat berikut sesuai dengan nama-nama dosen pembimbing
    Dosen Pembimbing   & : & 1. Dr. Supeno Mardi Susiki Nugroho, S.T, M.T.                \\
                       &   & 2. Reza Fuad Rachmadi, S.T., M.T., Ph.D                      \\
    % Ubah kalimat berikut sesuai dengan judul tugas akhir
    Judul Tugas Akhir  & : & \textbf{Sistem rekomendasi Mata Kuliah Pilihan dan Pengayaan
    berbasis Deep Learning}                                                               \\
    Uraian Tugas Akhir & : &                                                              \\
  \end{tabular}
\end{adjustwidth}

Mahasiswa yang sedang menempuh semester empat atau selanjutnya biasanya akan merencanakan
untuk mengambil mata kuliah pilihan maupun mata kuliah penganyaan. Permasalahan yang biasanya dihadapi oleh mahasiswa
saat akan melakukan pengambilan mata kuliah adalah kebingungan dalam memilih mata kuliah
yang sesuai minat dan kemampuan mereka. Banyak mahasiswa yang masih belum memiliki gambaran
tentang minat pada mata kuliah disemester selanjutnya. Oleh karena itu, sistem rekomendasi
sangat cocok untuk dapat membantu mahasiswa dalam memutuskan mata kuliah apa yang akan mereka
untuk ditempuh disemester selanjutnya. Oleh karena itu dibutuhkan sebuah sistem yang dapat memberikan
rekomendasi mata kuliah kepada mahasiswa sehingga dapat membantu mahasiswa membuat keputusan saat akan
mengambil mata kuliah.
\vspace{1ex}

\begin{flushright}
  Surabaya, 23 Oktober 2022
\end{flushright}
\vspace{1ex}

\begin{center}

  \begin{multicols}{2}

    Dosen Pembimbing 1
    \vspace{12ex}

    % Ubah kalimat-kalimat berikut sesuai dengan nama dan NIP dosen pembimbing kedua
    \underline{Dr. Supeno Mardi Susiki Nugroho, S.T, M.T.} \\
    NIP. 19700313 199512 1 001

    \columnbreak

    Dosen Pembimbing 2
    \vspace{12ex}

    % Ubah kalimat-kalimat berikut sesuai dengan nama dan NIP dosen pembimbing kedua
    \underline{Reza Fuad Rachmadi, S.T., M.T., Ph.D} \\
    NIP. 19850403201212 1 000

  \end{multicols}

  % Dosen Pembimbing \\
  % \vspace{12ex}
  % % Ubah kalimat-kalimat berikut sesuai dengan nama dan NIP dosen pembimbing pertama
  % \underline{Dr. Supeno Mardi Susiki Nugroho, S.T, M.T.} \\
  % NIP. 19700313 199512 1 001

  \vspace{6ex}

  Mengetahui, \\
  % Ubah kalimat berikut sesuai dengan jabatan kepala departemen
  Kepala Departemen Teknik Komputerr FTEIC - ITS
  \vspace{12ex}

  % Ubah kalimat-kalimat berikut sesuai dengan nama dan NIP kepala departemen
  \underline{Dr. Supeno Mardi Susiki Nugroho, S.T, M.T.} \\
  NIP. 19700313 199512 1 001

\end{center}
