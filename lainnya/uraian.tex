{\setstretch{1.15}
\noindent\textbf{Departemen Teknik Komputer - FTEIC}     \\
\textbf{Institut Teknologi Sepuluh Nopember}    \\

\begin{center}
    \large{\textbf{\underline{EC224701 PROPOSAL TUGAS AKHIR – 2 SKS}}}
\end{center}

\begin{adjustwidth}{-0.2cm}{}
    \begin{tabular}{lcp{0.7\linewidth}}
        \noindent{Nama Mahasiswa}     & : & {Abdul Azis}                                                                                                                                \\
        \noindent{Nomer Pokok}        & : & {0721 19 4000 7004}                                                                                                                         \\
        \noindent{Semester}           & : & {Ganjil 2022/2023}                                                                                                                          \\
        \noindent{Dosen Pembimbing}   & : & {1. Dr. Supeno Mardi Susiki Nugroho, S.T, M.T.}                                                                                             \\
                                      &   & {2. Reza Fuad Rachmadi, S.T., M.T., Ph.D.}                                                                                                  \\
        \noindent{Judul Tugas Akhir}  & : & \textbf{Rekomendasi mata kuliah pilihan dan pengayaan menggunakan \emph{Neural Collaborative Filtering} dan \emph{Content-Based Filtering}} \\
        \noindent{Uraian Tugas Akhir} & : &
    \end{tabular}
\end{adjustwidth}

\noindent{Tugas akhir ini akan mengembangkan sistem rekomendasi mata kuliah pilihan dan pengayaan di perguruan tinggi menggunakan
    \emph{Neural Collaborative Filtering} dan \emph{Content-based Filtering}. Metodologi yang akan digunakan meliputi explorasi dataset,
    pra-pemrosesan data, \emph{grades prediction}, \emph{similarity measure}, \emph{evaluasi model}, dan \emph{retrieval recommendations}. Tujuan
    utama dari tugas akhir ini adalah membantu mahasiswa dalam memilih mata kuliah pilihan dan pengayaan yang sesuai dengan minat dan kemampuan mereka.
    Hasil penelitian diharapkan dapat memberikan rekomendasi yang relevan untuk mahasiswa dan dapat menjadi referensi bagi perguruan tinggi dalam
    meningkatkan kualitas pengalaman belajar mahasiswa.}

\begin{center}

    \begin{multicols}{2}

        Dosen Pembimbing 1
        \vspace{12ex}

        \underline{Dr. Supeno Mardi Susiki Nugroho, S.T, M.T.} \\
        NIP. 19700313 199512 1 001

        \columnbreak

        Dosen Pembimbing 2
        \vspace{12ex}

        \underline{Reza Fuad Rachmadi, S.T., M.T., Ph.D.} \\
        NIP. 19850403 201212 1 001

    \end{multicols}
    \vspace{6ex}

    Mengetahui, \\
    Kepala Departemen Teknik Komputer FTEIC - ITS
    \vspace{12ex}

    \underline{Dr. Supeno Mardi Susiki Nugroho, S.T, M.T.} \\
    NIP. 19700313 199512 1 001

\end{center}
}

