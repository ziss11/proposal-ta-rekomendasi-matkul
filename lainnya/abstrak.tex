{\setstretch{1.15}
\begin{center}
    \uppercase{\textbf{\large Rekomendasi mata kuliah pilihan dan pengayaan menggunakan \emph{Neural Collaborative Filtering} dan \emph{Content-Based Filtering}}}
\end{center}

\vspace*{6 mm}
\begin{adjustwidth}{-0.2cm}{}
    \begin{tabular}{lcp{0.7\linewidth}}
        \noindent\textbf{Nama Mahasiswa / NRP} & : & \textbf{Abdul Azis / 0721 19 4000 7004}                \\
        \noindent\textbf{Departemen}           & : & \textbf{Teknik Komputer FTEIC - ITS}                   \\
        \noindent\textbf{Dosen Pembimbing}     & : & \textbf{1. Dr. Supeno Mardi Susiki Nugroho, S.T, M.T.} \\
                                               &   & \textbf{2. Reza Fuad Rachmadi, S.T., M.T., Ph.D.}      \\
    \end{tabular}
\end{adjustwidth}

\vspace{6 mm}
\noindent\textbf{Abstrak}
\vspace{3 mm}

Mahasiswa yang sedang menempuh semester empat atau selanjutnya biasanya akan merencanakan untuk mengambil mata kuliah pilihan maupun mata kuliah penganyaan.
Permasalahan yang biasanya dihadapi oleh mahasiswa saat akan melakukan pengambilan mata kuliah adalah kebingungan dalam memilih mata kuliah yang sesuai minat dan kemampuan
mereka. Banyak mahasiswa yang masih belum memiliki gambaran tentang minat pada mata kuliah disemester selanjutnya. Tak sedikit pula mahasiswa yang telah memiliki
perminatan tetapi masih bingung dalam memilih mata kuliah yang akan ditempuh. Oleh karena itu, sistem rekomendasi sangat cocok untuk dapat membantu
mahasiswa dalam memutuskan mata kuliah apa yang akan mereka untuk ditempuh disemester selanjutnya. Penelitian ini bertujuan untuk merancang dan membuat
sebuah sistem sehingga dapat memberikan rekomendasi mata kuliah pilihan maupun mata kuliah pengayaan yang mungkin cocok dengan mahasiswa, dan untuk
mengetahui kelayakan sistem rekomendasi yang dibangun dengan pendekatan pendekatan yang digunakan. Metode yang digunakan yaitu \emph{hybrid recommender system} dengan
menggunakan pendekatan \emph{collaborative filtering} dan \emph{content-based fitlering}. \emph{Collaborative filtering} bekerja dengan membuat suatu database yang berisi item preferences
dari pengguna. \emph{Collaborative filtering} mampu memberikan rekomendasi yang lebih dari sekedar kemiripan item, melainkan dapat memberikan rekomendasi dengan mempelajari selera pengguna dan
mencari pengguna yang memiliki selera yang kurang lebih sama. Sistem rekomendasi dengan metode \emph{content-based filtering} melakukan proses \emph{learning} untuk merekomendasikan
item yang mirip dengan item sebelumnya yang disukai atau dipilih oleh pengguna. Kemiripan item dihitung berdasarkan pada fitur-fitur yang ada pada item yang dibandingkan.
Oleh karenanya, metode ini tidak bergantung pada situasi apakah item tersebut merupakan item baru (yang belum pernah dipilih oleh pengguna manapun) maupun bukan item baru. Konten yang
menjadi obyek learning pada metode ini umumnya adalah informasi fitur mengenai item yang tersedia, namun pada perkembangannya, profil pengguna juga dapat menjadi content yang dipelajari
kemiripannya. Sistem ini akan dibangun dengan menggunakan \emph{Deep Learning} yang dapat lebih efisien dan lebih tepat sasaran, karena dapat menjalankan tugas-tugas yang kompleks.

\vspace{6 mm}
\noindent
\textbf{Kata Kunci: }Sistem Rekomendasi, \emph{Hybrid Recommender System}, \emph{Deep Learning}
}