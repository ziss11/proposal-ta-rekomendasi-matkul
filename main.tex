% Pengaturan ukuran teks dan jenis dokumen
\documentclass[12pt]{report}

% Pengaturan ukuran halaman dan margin
\usepackage[a4paper, top=30mm,left=30mm,right=20mm,bottom=25mm]{geometry}

% Pengaturan ukuran spasi
\usepackage[singlespacing]{setspace}

% Judul dokumen
\title{Proposal Tugas Akhir ITS}
\author{Abdul Azis}

% Pengaturan format bahasa
\usepackage[indonesian]{babel}

% Pengaturan detail pada file PDF
\usepackage[pdfauthor={\@author},bookmarksnumbered,pdfborder={0 0 0}]{hyperref}

% Pengaturan jenis karakter
\usepackage[utf8]{inputenc}

% Package lainnya
% \usepackage{mathptmx}
\usepackage{etoolbox} %s Mengubah fungsi default
\usepackage{enumitem} % Pembuatan list
\usepackage{lipsum} % Pembuatan template kalimat
\usepackage{graphicx} % Input gambar
\usepackage{longtable} % Pembuatan tabel
\usepackage[table,xcdraw]{xcolor} % Pewarnaan tabel
\usepackage[numbers]{natbib} % Kutipan artikel
\usepackage{changepage} % Pembuatan teks kolom
\usepackage{multicol} % Pembuatan kolom ganda
\usepackage{multirow} % Pembuatan baris ganda
\usepackage{eso-pic}
\usepackage{anyfontsize}

% Pengaturan penomoran halaman
\usepackage{fancyhdr}
\fancyhf{}
\renewcommand{\headrulewidth}{0pt}
\pagestyle{fancy}
\fancyfoot[C]{\thepage}
\patchcmd{\chapter}{plain}{fancy}{}{}
\patchcmd{\chapter}{empty}{plain}{}{}

% Pengaturan format judul bab
\usepackage{titlesec}
\renewcommand{\thechapter}{\Roman{chapter}}
\renewcommand{\thesection}{\arabic{chapter}.\arabic{section}}  
\renewcommand{\thetable}{\arabic{chapter}.\arabic{table}}  
\renewcommand{\thefigure}{\arabic{chapter}.\arabic{figure}}

\titleformat{\chapter}[display]{\bfseries\centering\Large}{BAB \thechapter}{8pt}{\MakeUppercase}
\titleformat*{\section}{\large\bfseries}
\titleformat*{\subsection}{\large\bfseries}
\titlespacing{\section}{0ex}{3ex}{1.5ex}
\titlespacing{\subsection}{0ex}{3ex}{1.5ex}
\titlespacing{\chapter}{0pt}{0pt}{20pt}

% Pengaturan daftar konten
\usepackage{titletoc}
\usepackage{etoolbox}
\usepackage[titles]{tocloft}

\makeatletter
\pretocmd{\chapter}{\addtocontents{toc}{\protect\addvspace{10\p@}}}{}{}
\pretocmd{\section}{\addtocontents{toc}{\protect\addvspace{5\p@}}}{}{}
\pretocmd{\section}{\addtocontents{toc}{\protect\addvspace{5\p@}}}{}{}
\makeatother

\renewcommand\cftfigafterpnum{\vskip5pt\par}
\renewcommand\cfttabafterpnum{\vskip5pt\par}
\renewcommand\cftbeforechapskip{\vskip5pt\par}

\titlecontents{chapter}
  [0pt]
  {\bfseries}
  {\MakeUppercase{\chaptername\ \thecontentslabel:\quad}}
  {}
  {\mdseries\titlerule*[0.75em]{.}\bfseries\contentspage}

% Isi keseluruhan dokumen
\begin{document}

% Sampul
\newcommand\covercontents{sampul/konten.tex}
\AddToShipoutPictureBG*{
    \AtPageLowerLeft{
        % Ubah nilai berikut jika posisi horizontal background tidak sesuai
        \hspace{-4mm}

        % Ubah nilai berikut jika posisi vertikal background tidak sesuai
        \raisebox{0mm}{
            \includegraphics[width=\paperwidth,height=\paperheight]{sampul/gambar/sampul.png}
        }
    }
}

% Menyembunyikan nomor halaman
\thispagestyle{empty}

% Pengaturan margin untuk menyesuaikan konten sampul
\newgeometry{
    top=65mm,
    left=30mm,
    right=30mm,
    bottom=20mm,
}

\begin{flushleft}

    % Pemilihan font sans serif
    \sffamily

    % Pemilihan font bold
    % \fontseries{bx}
    \selectfont

    \input{\covercontents}

\end{flushleft}

\restoregeometry

% Atur ulang penomoran halaman
\pagenumbering{roman}

% Lembar pengesahan
{\setstretch{1.15}
\begin{center}
    \uppercase{\textbf{\large Lembar Pengesahan}} \\

    \vspace*{10mm}
    \uppercase{\textbf{Sistem rekomendasi Mata Kuliah Pilihan dan Pengayaan berbasis Deep Learning}}
    \vspace*{10mm}

    {\setstretch{1.5}
        \uppercase{\textbf{Proposal Tugas Akhir}} \\
        Diajukan untuk memenuhi salah satu syarat          \\
        memperoleh gelar Sarjana pada                      \\
        Program Studi S-1 Teknik Komputer                  \\
        Departemen Teknik Komputer                         \\
        Fakultas Teknologi Elektro dan Informatika Cerdas  \\
        Institut Teknologi Sepuluh Nopember                \\
    }

    \vspace*{1.5cm}
    Oleh: \textbf{\uppercase{Abdul Azis}}                          \\
    NRP: \textbf{0721 19 4000 7004}                    \\
    \vspace*{1.5cm}

    Disetujui oleh Tim Penguji Proposal Tugas Akhir :  \\
    \vspace*{5mm}

    \begin{enumerate}
        \setlength\itemsep{1.5em}
        \item Dr. Supeno Mardi Susiki Nugroho, S.T, M.T. \hfill Pembimbing
        \item Reza Fuad Rachmadi, S.T., M.T., Ph.D. \hfill Ko-Pembimbing
        \item - \hfill -
        \item - \hfill -
        \item - \hfill -
    \end{enumerate}

    \vspace*{1.5cm}
    \uppercase{\textbf{surabaya}}                      \\
    \textbf{Desember, 2022}                              \\
\end{center}
}

\newpage

% Abstrak
{\setstretch{1.15}
\begin{center}
    \uppercase{\textbf{\large Sistem rekomendasi Mata Kuliah Pilihan dan Pengayaan berbasis Deep Learning}}
\end{center}

\vspace*{6 mm}
\begin{adjustwidth}{}{}
    \begin{tabular}{lcp{1\linewidth}}
        \noindent\textbf{Nama Mahasiswa / NRP} & : & \textbf{Abdul Azis / 0721 19 4000 7004}                \\
        \noindent\textbf{Departemen}           & : & \textbf{Teknik Komputer FTEIC - ITS}                   \\
        \noindent\textbf{Dosen Pembimbing}     & : & \textbf{1. Dr. Supeno Mardi Susiki Nugroho, S.T, M.T.} \\
                                               & : & \textbf{2. Reza Fuad Rachmadi, S.T., M.T., Ph.D.}      \\
    \end{tabular}
\end{adjustwidth}

\vspace{6 mm}
\noindent
\textbf{Abstrak}
\vspace{3 mm}

Mahasiswa yang sedang menempuh semester empat atau selanjutnya biasanya akan merencanakan untuk mengambil mata kuliah pilihan maupun mata kuliah penganyaan.
Permasalahan yang biasanya dihadapi oleh mahasiswa saat akan melakukan pengambilan mata kuliah adalah kebingungan dalam memilih mata kuliah yang sesuai minat dan kemampuan
mereka. Banyak mahasiswa yang masih belum memiliki gambaran tentang minat pada mata kuliah disemester selanjutnya. Tak sedikit pula mahasiswa yang telah memiliki
perminatan tetapi masih bingung dalam memilih mata kuliah yang akan ditempuh. Untuk dapat membantu mahasiswa dalam memutuskan pengambilan mata kuliah diperlukan sebuah
sistem rekomendasi yang dapat memberikan sebuah rekomendasi mata kuliah. Penelitian dilakukan dengan mencari kemiripan antar mata kuliah dengan menggunakan pendekatan \emph{hybrid recommender system} yang menggabungkan dua
pendekatan sistem rekomendasi yaitu \emph{collaborative filtering} dan \emph{content-based filtering} yang mana dengan menggabungkan dua pendekatan tersebut akan saling menutupi
kekurangan dari masing-masing pendekatan dalam memberikan rekomendasi dan menggunakan \emph{Deep Learning} untuk membangun model berbasis pendekatan \emph{hybrid recommender system}. Sehingga, penelitian ini
bertujuan untuk mengetahui kelayakan sistem rekomendasi yang dibangun dengan pendekatan yang digunakan.

\vspace{6 mm}
\noindent
\textbf{Kata Kunci: }Recommendation System, \emph{Hybrid Recommender System}, \emph{Deep Learning}
}
\addcontentsline{toc}{chapter}{ABSTRAK}
\newpage

% Daftar isi
\renewcommand*\contentsname{DAFTAR ISI}
\tableofcontents
\addcontentsline{toc}{chapter}{\contentsname}
\newpage

% Daftar gambar
\renewcommand*\listfigurename{DAFTAR GAMBAR}
\listoffigures
\addcontentsline{toc}{chapter}{\listfigurename}
\newpage

% Daftar tabel
\renewcommand*\listtablename{DAFTAR TABEL}
\listoftables
\addcontentsline{toc}{chapter}{\listtablename}
\newpage

% Nomor halaman isi dimulai dari sini
\pagenumbering{arabic}

% Konten pendahuluan
\section{PENDAHULUAN}

\subsection{Latar Belakang}

Mahasiswa yang sedang menempuh semester empat atau selanjutnya biasanya akan merencanakan untuk mengambil
mata kuliah pilihan maupun mata kuliah penganyaan. Proses ini biasanya dilakukan diawal semester sebelum proses perkuliahan
dimulai untuk mengambil mata kuliah yang akan ditempuh disemester yang baru. Dalam suatu perguruan tinggi, mahasiswa belajar dalam suatu
sistem kurikulum yang terdiri atas sekumpulan daftar mata kuliah dimana setelah menempuh mata kuliah,
mahasiswa akan mendapatkan nilai. Nilai mata kuliah yang akan diterima didasarkan atas kemampuan
mahasiswa.

Permasalahan yang biasanya dihadapi oleh mahasiswa saat akan melakukan pengambilan mata kuliah adalah kebingungan dalam
memilih mata kuliah yang sesuai minat dan kemampuan mereka. Banyak mahasiswa yang masih belum memiliki gambaran tentang
minat pada mata kuliah disemester selanjutnya. Tak sedikit pula mahasiswa yang telah memiliki perminatan tetapi masih
bingung dalam memilih mata kuliah yang akan ditempuh. Hal ini biasanya terjadi karena kurangnya pemahaman mengenai
keterkaitan antar mata kuliah yang telah, sedang dan akan ditempuh. Oleh karena itu, sistem rekomendasi sangat cocok untuk
dapat membantu mahasiswa dalam memutuskan mata kuliah apa yang akan mereka untuk ditempuh disemester selanjutnya.

Sistem rekomendasi adalah sistem yang bertujuan untuk memberikan rekomendasi item yang menarik bagi pengguna agar pengguna
dapat memutuskan item apa yang akan mereka pilih. Belakangan ini, sistem rekomendasi telah menjadi sangat populer dan menjadi
sebuah bagian yang penting diberbagai layanan. Salah satu teknik yang dapat digunakan untuk mengatasi permasalahan tersebut adalah
penerapan machine learning, khususnya metode klasifikasi untuk membantu dalam mengambil keputusan dengan menggunakan algoritma
\emph{Decision Tree}.

\subsection{Rumusan Masalah}
Berdasarkan uraian latar belakang diatas dapat dirumuskan permasalahan sebagai berikut:
\begin{enumerate}[noitemsep]
      \item Bagaimana sistem rekomendasi dapat memberikan rekomendasi mata kuliah pilihan dan pengayaan
            yang cocok dengan mahasiswa?
      \item Bagaimana mengetahui kelayakan sistem rekomendasi yang dibangun dari algoritma dan pendekatan
            digunakan?
\end{enumerate}

\subsection{Batasan Masalah}
Batasan-batasan permasalahan yang dijadikan sebagai pedoman dalam pelaksanaan penelitian ini adalah sebagai berikut:
\begin{enumerate}[noitemsep]
      \item Sistem rekomendasi ini pemilihan mata kuliah
            ini hanya digunakan sebagai rekomendasi bagi
            mahasiswa, keputusan pemilihan mata kuliah
            ada kepada mahasiswa.
      \item Metode yang digunakan adalah item-based
            collaborative filtering dan content-based
            filtering dengan algoritma Decision Tree.
      \item Data latih yang digunakan berupa data mahasiswa, data mata kuliah, dan data nilai mahasiswa yang ada di ITS.
\end{enumerate}

\subsection{Tujuan Penelitian}
Tujuan penelitian ini dilakukan adalah:
\begin{enumerate}[noitemsep]
      \item Untuk merancang dan membuat sebuah sistem sehingga dapat memberikan rekomendasi mata kuliah pilihan
            maupun mata kuliah pengayaan yang mungkin cocok dengan mahasiswa.
      \item Untuk mengetahui kelayakan sistem rekomendasi yang dibangun dari algoritma dan pendekatan
            digunakan.
\end{enumerate}

\subsection{Manfaat}
Hasil penelitian ini yang diharapkan agar dapat memberikan manfaat sebagai berikut:
\begin{enumerate}[noitemsep]
      \item Bagi penulis yaitu untuk menambahkan pengetahuan dalam membangun sebuah sistem
            otomatisasi dengan menggunakan machine learning berbasis sistem rekomendasi dengan pendekatan hybrid recommender system menggunakan content-based filtering dan collaborative filtering
      \item Memberikan manfaat bagi pengembangan ilmu pengetahuan dan teknologi serta wawasan ilmu bagi
            penelitian dan pengembangan selanjutnya.
      \item Memberikan kemudahan bagi mahasiswa agar dapat terbantu dengan adanya sistem yang dapat memberikan rekomendasi matakuliah yang mungkin cocok dengan preferensi mereka.
\end{enumerate}
\newpage

% Konten tinjauan pustaka
\section{TINJAUAN PUSTAKA}

\subsection{Penelitian Terdahulu}
Penelitian mengenai rekomendasi mata kuliah telah di lakukan. Pada tahun 2019 Glenn Ferio, Rolly Intan, dan Silvia Rostianingsih melakukan
penelitian mengenai sistem rekomendasi mata kuliah pilihan pada \emph{paper} yang berjudul
\emph{Sistem Rekomendasi Mata Kuliah Pilihan Menggunakan Metode pengguna Based Collaborative Filtering Berbasis Algoritma Adjusted Cosine Similarity}.
Input yang digunakan dalam penelitian ini adalah nilai-nilai mata kuliah yang didapatkan oleh mahasiswa. Pada penelitian ini mereka menggunakan algoritma
\emph{Cosine Similarity}, \emph{Adjusted Cosine Similarity}, dan model \emph{K-Nearest Neighbors}. Pada penelitian tersebut mereka mendapatkan akurasi
pada data \emph{testing} sebesar 69.24\% untuk algoritma \emph{Cosine Similarity} dan 83.55\% untuk algoritma \emph{Adjusted Cosine Similarity}.

\begin{table} [ht] \centering
  \counterwithin{table}{section}
  \caption{Hasil akurasi algoritma \emph{Cosine Similarity} dan \emph{Adjusted Cosine Similarity} \citep{cosineSimilarity}}
  \vspace*{3mm}
  \includegraphics{gambar/akurasi-algoritma-cosine-similarity.png}
\end{table}

Dan mendapatkan akurasi pada data \emph{testing} menggunakan model \emph{K-Nearest Neighbors} dengan jumlah K = 16 sebesar 89.31\%
dan mendapatkan Mean Absolute Error sekitar 10.75.

\begin{figure} [ht] \centering
  \counterwithin{figure}{section}
  \includegraphics{gambar/mae-knn.png}
  \vspace*{3mm}
  \caption{Visualisasi Mean Absolute Error model \emph{K-Nearest Neighbors} \citep{cosineSimilarity}}
\end{figure}


Pada tahun 2022 Edward Fernando, Panca Mudjiraharjo, dan Muhammad Aswin melakukan penelitian untuk membuat sistem rekomendasi
dengan pendekatan \emph{Collacorative filtering} dan algoritma \emph{K-Means Clustering}. Dalam penelitian ini, sistem rekomendasi
berbasis collaborative filtering dibangun dengan empat metode, yaitu \emph{jaccard distance}, \emph{euclidean distance}, \emph{cosine similiarity}, dan \emph{pearson correlation}.
Dari pengimplementasian keempat metode tersebut, sistem mampu melakukan prediksi terkait konsentrasi prodi mahasiswa dan juga rekomendasi mata kuliah
yang dapat diambil pada semester selanjutnya. Dari implementasi awal, didapatkan hasil bahwa metode yang paling optimal hingga yang kurang optimal
secara berurutan adalah metode \emph{cosine similiarity}, \emph{euclidean distance}, \emph{jaccard distance}, lalu \emph{pearson correlation}.

\begin{table} [ht] \centering
  \counterwithin{table}{section}
  \caption{Nilai Kenaikan Akurasi Pendekatan \emph{Collaborative Filtering} dengan \emph{K-Means Clustering} \citep{clustering}}
  \vspace*{3mm}
  \begin{tabular}{|c|cc|c|}
    \hline
    \multirow{2}{*}{Metode} & \multicolumn{2}{c|}{\begin{tabular}[c]{@{}c@{}}Moving Average Akurasi \\ MK\end{tabular}} & \multirow{2}{*}{Kenaikan Akurasi}                                  \\ \cline{2-3}
                            & \multicolumn{1}{c|}{\begin{tabular}[c]{@{}c@{}}Sebelum \\ K-means\end{tabular}}           & \begin{tabular}[c]{@{}c@{}}Sesusah\\ K-Means\end{tabular} &        \\ \hline
    Jaccard                 & \multicolumn{1}{c|}{51.18\%}                                                              & 53.66\%                                                   & 4.83\% \\ \hline
    Euclidean Distance      & \multicolumn{1}{c|}{56.44\%}                                                              & 62.04\%                                                   & 9.92\% \\ \hline
    Cosine Similiarity      & \multicolumn{1}{c|}{59.25\%}                                                              & 64.76\%                                                   & 9.30\% \\ \hline
    Pearson Correlation     & \multicolumn{1}{c|}{40.83\%}                                                              & 43.71\%                                                   & 7.05\% \\ \hline
  \end{tabular}
\end{table}

Proses optimalisasi hasil rekomendasi mata kuliah pada penelitian ini dilakukan dengan melakukan seleksi rekomendasi berdasarkan tabel keputusan
yang dioleh menggunakan metode \emph{K-Means Clustering}. \emph{K-Means Clustering} mampu melakukan klasterisasi dari seluruh mata kuliah yang muncul dalam rekomendasi
dan memberikan label untuk mana mata kuliah yang direkomendasikan dan tidak direkomendasikan. Pada pengimplementasiannya, tabel keputusan \emph{K-Means Clustering}
mampu meningkatkan nilai akurasi \emph{moving average} pada setiap metode pendekatan \emph{collaborative filtering} dengan persentase peningkatan akurasi secara berurutan
dari yang paling optimal hingga kurang optimal yaitu \emph{euclidean distance} sebesar 9.92\%, \emph{cosine similiarity} sebesar 9.30\%, \emph{pearson correlation} sebesar 7.05\%,
dan \emph{jaccard} sebesar 4.83\%. Secara keseluruhan, sistem rekomendasi mata kuliah ini mampu melakukan prediksi konsentrasi mahasiswa hingga 84.78\% dan akurasi
\emph{moving average} pada prediksi rekomendasi mata kuliah hingga 64.76\% menggunakan data latih yang ditentukan. Dengan nilai \emph{mean absolute error} yang dihasilkan
pada penelitian ini, pendekatan \emph{collaborative filtering} yang paling optimal untuk perekomendasian mata kuliah jatuh pada metode \emph{cosine similiarity}.

\begin{table} [ht] \centering
  \counterwithin{table}{section}
  \caption{Nilai Mean Absolute Error Akurasi Mata Kuliah \citep{clustering}}
  \vspace*{3mm}
  \begin{tabular}{|c|cc|c|}
    \hline
    \multirow{2}{*}{Metode} & \multicolumn{2}{c|}{\begin{tabular}[c]{@{}c@{}}Moving Average Akurasi \\ MK\end{tabular}} & \multirow{2}{*}{Kenaikan Akurasi}                                  \\ \cline{2-3}
                            & \multicolumn{1}{c|}{\begin{tabular}[c]{@{}c@{}}Sebelum \\ K-means\end{tabular}}           & \begin{tabular}[c]{@{}c@{}}Sesusah\\ K-Means\end{tabular} &        \\ \hline
    Jaccard                 & \multicolumn{1}{c|}{51.18\%}                                                              & 53.66\%                                                   & 4.83\% \\ \hline
    Euclidean Distance      & \multicolumn{1}{c|}{56.44\%}                                                              & 62.04\%                                                   & 9.92\% \\ \hline
    Cosine Similiarity      & \multicolumn{1}{c|}{59.25\%}                                                              & 64.76\%                                                   & 9.30\% \\ \hline
    Pearson Correlation     & \multicolumn{1}{c|}{40.83\%}                                                              & 43.71\%                                                   & 7.05\% \\ \hline
  \end{tabular}
\end{table}

\subsection{Machine Learning}
\emph{Machine Learning} sejenis kecerdasan buatan yang memungkinkan perangkat lunak untuk menjadi lebih akurat dalam memprediksi keluaran
tanpa harus memprogram secara eksplisit untuk melakukannya. Algoritma \emph{machine learning} menggunakan data riwayat sebagai input untuk
memprediksi keluaran baru atau menemukan pola pada data.

\subsection{Sistem Rekomendasi}
Sistem rekomendasi merupakan suatu sistem yang dapat memberikan informasi dan rekomendasi yang membantu pengguna dalam membuat keputusan berdasarkan data yang telah ada sebelumnya
  {(Edward, Panca, \& Aswin, 2022)}. Sistem rekomendasi akan memberikan rekomendasi yang berbeda kepada setiap pengguna, bukan sekedar memberikan daftar item paling banyak diminati, melainkan memberikan
saran mengenai item-item yang mungkin sesuai untuk pengguna. Artinya, setiap pengguna akan mendapatkan rekomendasi yang berbeda, sesuai dengan profil dan minat
pengguna tersebut \citep{contentBasedXCollaborative}.

Sistem ini tidak merekomendasikan item secara spesifik yang pengguna inginkan. Sistem akan merekomendasikan sejumlah item yang mungkin cocok, jadi jangan heran jika
item yang direkomendasikan oleh sistem terkadang tidak sesuai dengan preferensi pengguna. Pada sistem rekomendasi, keluarannya adalah rekomendasi “top-N”. Sistem akan memberikan sejumlah
rekomendasi peringkat teratas yang cocok dengan pengguna.

Meskipun demikian, penentuan rekomendasi personal mensyaratkan bahwa sistem harus memiliki pengetahuan tentang pengguna. Setiap system rekomendasi harus membuat dan
memaintain suatu model pengguna atau profil pengguna yang, misalnya, memuat informasi mengenai minat atau preferences dari pengguna \citep{contentBasedXCollaborative}.

\subsection{Content-based Filtering}
\emph{Content-based Filtering} adalah jenis sistem pemberi rekomendasi yang mencoba menebak apa yang mungkin disukai pengguna berdasarkan aktivitas pengguna tersebut. \emph{Content-based Filtering}
membuat rekomendasi dengan menggunakan kata kunci dan atribut yang ditetapkan ke objek dalam database (misalnya, item di pasar online) dan mencocokkannya dengan profil pengguna.
Sistem rekomendasi dengan metode \emph{content-based filtering} melakukan proses learning untuk merekomendasikan item yang mirip dengan item sebelumnya
yang disukai atau dipilih oleh pengguna. Kemiripan item dihitung berdasarkan pada fitur-fitur yang ada pada item yang dibandingkan. Oleh karenanya, metode ini tidak bergantung
pada situasi apakah item tersebut merupakan item baru {(yang belum pernah dipilih oleh pengguna manapun)} maupun bukan item baru. Konten yang menjadi obyek \emph{learning} pada
metode ini umumnya adalah informasi fitur mengenai item yang tersedia, namun pada perkembangannya, profil pengguna juga dapat menjadi content yang dipelajari
kemiripannya \citep{contentBasedXCollaborative}.

Sistem akan memilih dan melakukan peringkat item berdasarkan kesamaan profil pengguna dan profil item.Kesamaan antara representasi dari pengguna dan representasi dari
item akan didasarkan pada prinsip kedekatan yang menyatakan bahwa jarak dari dua deskripsi item secara langsung berkaitan dengan kesamaan mereka (Putri, 2019).
Misalnya, jika pengguna memiliki positif menilai film yang termasuk dalam genre komedi, maka sistem dapat belajar untuk
merekomendasikan film lain dari genre ini \citep{handbook}.


\subsection{Collaborative Filtering}
\emph{Collaborative Filtering} adalah suatu konsep dimana opini dari pengguna lain yang ada digunakan untuk memprediksi item yang mungkin disukai/diminati oleh seorang pengguna \citep{handbook}.
\emph{Collaborative filtering} bekerja dengan membuat suatu \emph{database} yang berisi item \emph{preferences} dari pengguna. Sebagai contoh, pengguna A dicocokkan terhadap data yang ada di \emph{database} untuk menemukan
\emph{neighbor}, yaitu pengguna lain yang memiliki kemiripan \emph{preference} dengan pengguna A. Dari kecocokan yang didapatkan, pengguna A akan diberikan rekomendasi item yang sebelumnya pernah dipilih oleh neighbor dari
pengguna A. \emph{Collaborative filtering} mampu memberikan rekomendasi yang lebih dari sekedar kemiripan item, melainkan dapat memberikan rekomendasi dengan mempelajari selera pengguna dan mencari pengguna yang memiliki selera yang kurang lebih sama.{(Aditya, Wayan, \& Budi, 2015)}.

Pada implementasi \emph{Collaborative filtering} dalam sistem rekomendasi mata kuliah pilihan, data yang dibandingkan akan disimpan dalam sebuah matriks dua dimensi dimana dimensi dari y atau baris akan menyatakan mahasiswa dan dimensi x atau kolom akan menyatakan mata kuliah
\citep{cosineSimilarity}.


\subsection{Decision Tree}
Decision tree merupakan algoritma \emph{supervised learning} yang dapat melakukan tugas klasifikasi maupun regresi. Algoritma ini merupakan sebuah metode kecerdasan buatan yang berbentuk pohon dimana pada setiap cabang memiliki
pilihan serta alternatif dengan daun yang menyatakan keputusan yang dipilih. \emph{Decision tree} banyak digunakan karena dapat menggabungkan sebuah pola atau informasi kedalam sebuah bentu pohon keputusan {(Dwita, Agus \& Rizki 2022)}.
\emph{Decision Tree} digunakan untuk mempelajari klasifikasi dan prediksi pola dari data dan menggambarkan relasi dari variabel attribut x dan variabel target y dalam bentuk pohon {(Ye, 2014)}. Pada umumnya, decision tree classifiers
melakukan analisa dan konversi data menjadi decision tree dan sekumpulan aturan \citep{contentBasedXCollaborative}.


\newpage

% Konten metodologi
\chapter{METODOLOGI}

\section{Metode yang digunakan}

Penelitian ini dilaksanakan sesuai dengan blok diagram pada Gambar 3.1. Blok diagram tersebut
merupakan metodologi penelitian yang disusun sesuai dengan langkah-langkah yang dilakukan dalam penelitian ini.

\begin{figure} [ht] \centering
  \includegraphics[width=160mm]{gambar/diagram-blok.png}
  \caption{Metodologi Penelitian}
\end{figure}

\section{Bahan dan peralatan yang digunakan}
\begin{enumerate}
  \item Komputer/Laptop
  \item Python
  \item Jupyter Notebook
  \item Tensorflow
  \item Docker
  \item Flask
\end{enumerate}

\section{Urutan pelaksanaan penelitian}

\begin{adjustwidth}{1em}{0pt}
  \addtocontents{toc}{\protect\setcounter{tocdepth}{1}}

  \subsection{Explorasi Dataset}
  Melakukan analisa pada data untuk mendapatkan gambaran awal pada data. Analisa yang dilakukan yaitu
  memeriksa informasi pada data (tipe data pada tiap data), memeriksa \emph{missing value} {(data yang hilang pada baris data)},
  memeriksa duplikasi pada data.

  \subsection{Pre-pemrosesan Data}
  Pada proses ini dataset yang telah dilakukan analisa akan dilakukan pra-pemrosesan sebelum dataset dapat digunakan
  untuk melakukan pelatihan pada model. Pada sistem rekomendasi pra-pemrosesan data yang biasa dilakukan seperti menghapus data
  yang tidak diperlukan untuk proses pelatihan, menghapus \emph{missing value} jika ada, melakukan normalisasi pada data (biasanya mentransformasikan data kedalam range yang sama),
  melakukan pembagian data pelatihan dan data percobaan.

  \subsection{\emph{Grades Prediction}}
  Untuk dapat menemukan mata kuliah yang mungkin sesuai dengan minat mahasiswa, diperlukan model yang dapat memprediksi nilai mata kuliah yang kemungkinan akan diambil oleh mahasiswa. Ini
  merupakan salah satu tahap dalam membangun sistem rekomendasi dengan pendekatan \emph{collaborative filtering} yang mana model akan memprediksi nilai yang akan didapatkan oleh mahasiswa dari berbagai mata kuliah
  yang belum pernah diambil sebelumnya oleh mahasiswa. Nilai yang diprediksi tidak hanya berdasarkan dari riwayat pengambilan mata kuliah dari satu orang mahasiswa, namun akan dilakukan pencocokan dengan mahasiwa
  lain yang memiliki preferensi yang saama. Setelah memprediksi nilai dari sekumpulan mata kuliah nantinya akan diambil \emph{top-N} mata kuliah dengan prediksi nilai tertinggi, setelah itu mata kuliah tersebut dapat direkomendasikan
  kepada mahasiswa.

  \subsection{\emph{Similarity Measure}}
  Tahap ini merupakan penerapan untuk pendekatan \emph{content-based filtering} yang mana model akan mencari kemiripan dari suatu mata kuliah, berdasarkan karakteristik yang dimiliki oleh mata kuliah tersebut.
  Setelah mendapatkan mendapatkan daftar mata kuliah yang mirip dengan input mata kuliah yang diberikan, hasil tersebut akan dikombinasikan dengan hasil rekomendasi dari model dengan pendekatan \emph{collaborative filtering}
  sehingga menghasilkan sistem rekomendasi dengan pendekatan \emph{hybrid recommender syste}.

  \subsection{Evaluasi Model}
  Evaluasi Model dilakukan dengan melakukan rekomendasi pada data percobaan. Jika dirasa rekomendasi belum cukup baik. Maka perlu melakukan \emph{tuning parameter} pada model seperti unit layer,
  jumlah layer pada model \emph{Deep Learning}, jumlah iterasi pelatihan dan beberapa parameter lain hingga model dapat memberikan rekomendasi dengan cukup baik.

  \subsection{\emph{Retrieval Recommendations}}
  Pada tahap ini model dari sistem rekomendasi yang telah dibuat sebelumnya ada dideploy atau dirilis agar dapat digunakan untuk memberikan rekomendasi melalui sebuah \emph{software}.
  \emph{Software} yang digunakan untuk menampilkan rekomendasi dari mata kuliah yang diberikan oleh model tersebut berupa website sederhana yang bisa menampilkan daftar mata kuliah
  yang direkomendasikan sesuai dengan ID/NRP dari mahasiswa.

  \begin{figure} [ht] \centering
    \includegraphics[width=150mm]{gambar/mockup-1.png}
    \includegraphics[width=150mm]{gambar/mockup-2.png}
    \caption{Mock Up Tampilan Website Sederhana}
  \end{figure}

\end{adjustwidth}




\newpage

% Konten lainnya
% \chapter{Hasil yang diharapkan}

\section{Hasil yang Diharapkan dari Penelitian}
Dari penelitian yang akan dilakukan, diharapkan dapat menghasilkan sebuah sistem yang mampu digunakan untuk
memberikan sebuah rekomendasi mata kuliah, sebagai alat bantu untuk mahasiswa dalam membuat keputusan. Dari penelitian ini juga
diharapkan pengembangan sistem rekomendasi ini dapat mempermudah mahasiswa menemukan mata kuliah yang sesuai minat dan kemampuan mereka,
khususnya mata kuliah pilihan dan mata kuliah pengayaan, sehingga dapat meminimalkan resiko mendapatkan nilai yang kurang memuaskan saat
menempuh mata kuliah yang dipilih.

\section{Hasil Pendahuluan}

% \section{RENCANA KERJA}

% % Ubah tabel berikut sesuai dengan isi dari rencana kerja
% \newcommand{\w}{}
% \newcommand{\G}{\cellcolor{gray}}
% \begin{table}[h!]
%   \begin{tabular}{|p{3.5cm}|c|c|c|c|c|c|c|c|c|c|c|c|c|c|c|c|}

%     \hline
%     \multirow{2}{*}{Kegiatan} & \multicolumn{16}{|c|}{Minggu} \\
%     \cline{2-17} &
%     1 & 2 & 3 & 4 & 5 & 6 & 7 & 8 & 9 & 10 & 11 & 12 & 13 & 14 & 15 & 16 \\
%     \hline

%     % Gunakan \G untuk mengisi sel dan \w untuk mengosongkan sel
%     Pengambilan data &
%     \G & \G & \G & \G & \w & \w & \w & \w & \w & \w & \w & \w & \w & \w & \w & \w \\
%     \hline

%     Pengolahan data &
%     \w & \w & \w & \w & \G & \G & \G & \G & \w & \w & \w & \w & \w & \w & \w & \w \\
%     \hline

%     Analisa data &
%     \w & \w & \w & \w & \w & \w & \w & \w & \G & \G & \G & \G & \w & \w & \w & \w \\
%     \hline

%     Evaluasi penelitian &
%     \w & \w & \w & \w & \w & \w & \w & \w & \w & \w & \w & \w & \G & \G & \G & \G \\
%     \hline

%   \end{tabular}
% \end{table}


% Daftar pustaka
\renewcommand{\bibsection}{}
\chapter*{DAFTAR PUSTAKA}
\addcontentsline{toc}{chapter}{DAFTAR PUSTAKA}
\vspace{-2ex}
\bibliographystyle{unsrt}
\bibliography{pustaka/pustaka.bib}
\end{document}
