\section{TINJAUAN PUSTAKA}

\subsection{Penelitian Terdahulu}

Penelitian mengenai rekomendasi mata kuliah telah di lakukan. Pada tahun 2019 Glenn Ferio, Rolly Intan, dan Silvia Rostianingsih melakukan
penelitian mengenai sistem rekomendasi mata kuliah pilihan pada \emph{paper} yang berjudul
\emph{Sistem Rekomendasi Mata Kuliah Pilihan Menggunakan Metode User Based Collaborative Filtering Berbasis Algoritma Adjusted Cosine Similarity}.
Input yang digunakan dalam penelitian ini adalah nilai-nilai mata kuliah yang didapatkan oleh mahasiswa. Pada penelitian ini mereka menggunakan algoritma
\emph{Cosine Similarity}, \emph{Adjusted Cosine Similarity}, dan model \emph{K-Nearest Neighbors}. Pada penelitian tersebut mereka mendapatkan akurasi
pada data \emph{testing} sebesar 69.24\% untuk algoritma \emph{Cosine Similarity} dan 83.55\% untuk algoritma \emph{Adjusted Cosine Similarity}.

\begin{table} [ht]
  \centering
  \counterwithin{table}{section}
  \caption{Hasil akurasi algoritma \emph{Cosine Similarity} dan \emph{Adjusted Cosine Similarity}}
  \vspace*{3mm}
  \includegraphics{gambar/akurasi-algoritma-cosine-similarity.png}
\end{table}

Dan mendapatkan akurasi pada data \emph{testing} menggunakan model \emph{K-Nearest Neighbors} dengan jumlah K = 16 sebesar 89.31\%
dan mendapatkan Mean Absolute Error sekitar 10.75.

\begin{figure} [ht]
  \centering
  \counterwithin{figure}{section}
  \caption{Visualisasi Mean Absolute Error model \emph{K-Nearest Neighbors}}
  \vspace*{3mm}
  \includegraphics{gambar/mae-knn.png}
\end{figure}

Pada tahun 2022 Edward Fernando, Panca Mudjiraharjo, dan Muhammad Aswin melakukan penelitian untuk membuat sistem rekomendasi
dengan pendekatan \emph{Collacorative filtering} dan algoritma \emph{K-Means Clustering}. Dalam penelitian ini, sistem rekomendasi
berbasis collaborative filtering dibangun dengan empat metode, yaitu \emph{jaccard distance}, \emph{euclidean distance}, \emph{cosine similiarity}, dan \emph{pearson correlation}.
Dari pengimplementasian keempat metode tersebut, sistem mampu melakukan prediksi terkait konsentrasi prodi mahasiswa dan juga rekomendasi mata kuliah
yang dapat diambil pada semester selanjutnya. Dari implementasi awal, didapatkan hasil bahwa metode yang paling optimal hingga yang kurang optimal
secara berurutan adalah metode \emph{cosine similiarity}, \emph{euclidean distance}, \emph{jaccard distance}, lalu \emph{pearson correlatio}n.

\begin{table} [ht]
  \centering
  \counterwithin{table}{section}
  \caption{Nilai Kenaikan Akurasi Pendekatan \emph{Collaborative Filtering} dengan \emph{K-Means Clustering}}
  \vspace*{3mm}
  \begin{tabular}{|c|cc|c|}
    \hline
    \multirow{2}{*}{Metode} & \multicolumn{2}{c|}{\begin{tabular}[c]{@{}c@{}}Moving Average Akurasi \\ MK\end{tabular}} & \multirow{2}{*}{Kenaikan Akurasi}                                  \\ \cline{2-3}
                            & \multicolumn{1}{c|}{\begin{tabular}[c]{@{}c@{}}Sebelum \\ K-means\end{tabular}}           & \begin{tabular}[c]{@{}c@{}}Sesusah\\ K-Means\end{tabular} &        \\ \hline
    Jaccard                 & \multicolumn{1}{c|}{51.18\%}                                                              & 53.66\%                                                   & 4.83\% \\ \hline
    Euclidean Distance      & \multicolumn{1}{c|}{56.44\%}                                                              & 62.04\%                                                   & 9.92\% \\ \hline
    Cosine Similiarity      & \multicolumn{1}{c|}{59.25\%}                                                              & 64.76\%                                                   & 9.30\% \\ \hline
    Pearson Correlation     & \multicolumn{1}{c|}{40.83\%}                                                              & 43.71\%                                                   & 7.05\% \\ \hline
  \end{tabular}
\end{table}

Proses optimalisasi hasil rekomendasi mata kuliah pada penelitian ini dilakukan dengan melakukan seleksi rekomendasi berdasarkan tabel keputusan
yang dioleh menggunakan metode \emph{K-Means Clustering}. \emph{K-Means Clustering} mampu melakukan klasterisasi dari seluruh mata kuliah yang muncul dalam rekomendasi
dan memberikan label untuk mana mata kuliah yang direkomendasikan dan tidak direkomendasikan. Pada pengimplementasiannya, tabel keputusan \emph{K-Means Clustering}
mampu meningkatkan nilai akurasi \emph{moving average} pada setiap metode pendekatan \emph{collaborative filtering} dengan persentase peningkatan akurasi secara berurutan
dari yang paling optimal hingga kurang optimal yaitu \emph{euclidean distance} sebesar 9.92\%, \emph{cosine similiarity} sebesar 9.30\%, \emph{pearson correlation} sebesar 7.05\%,
dan \emph{jaccard} sebesar 4.83\%. Secara keseluruhan, sistem rekomendasi mata kuliah ini mampu melakukan prediksi konsentrasi mahasiswa hingga 84.78\% dan akurasi
\emph{moving average} pada prediksi rekomendasi mata kuliah hingga 64.76\% menggunakan data latih yang ditentukan. Dengan nilai \emph{mean absolute error} yang dihasilkan
pada penelitian ini, pendekatan \emph{collaborative filtering} yang paling optimal untuk perekomendasian mata kuliah jatuh pada metode \emph{cosine similiarity}.

\begin{table} [ht]
  \centering
  \counterwithin{table}{section}
  \caption{Nilai Mean Absolute Error Akurasi Mata Kuliah}
  \vspace*{3mm}
  \begin{tabular}{|c|cc|c|}
    \hline
    \multirow{2}{*}{Metode} & \multicolumn{2}{c|}{\begin{tabular}[c]{@{}c@{}}Moving Average Akurasi \\ MK\end{tabular}} & \multirow{2}{*}{Kenaikan Akurasi}                                  \\ \cline{2-3}
                            & \multicolumn{1}{c|}{\begin{tabular}[c]{@{}c@{}}Sebelum \\ K-means\end{tabular}}           & \begin{tabular}[c]{@{}c@{}}Sesusah\\ K-Means\end{tabular} &        \\ \hline
    Jaccard                 & \multicolumn{1}{c|}{51.18\%}                                                              & 53.66\%                                                   & 4.83\% \\ \hline
    Euclidean Distance      & \multicolumn{1}{c|}{56.44\%}                                                              & 62.04\%                                                   & 9.92\% \\ \hline
    Cosine Similiarity      & \multicolumn{1}{c|}{59.25\%}                                                              & 64.76\%                                                   & 9.30\% \\ \hline
    Pearson Correlation     & \multicolumn{1}{c|}{40.83\%}                                                              & 43.71\%                                                   & 7.05\% \\ \hline
  \end{tabular}
\end{table}

\subsection{Sistem Rekomendasi}



