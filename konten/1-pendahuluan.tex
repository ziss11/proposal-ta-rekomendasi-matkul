\section{PENDAHULUAN}

\subsection{Latar Belakang}

Mahasiswa yang sedang menempuh semester empat atau selanjutnya biasanya akan merencanakan untuk mengambil
mata kuliah pilihan maupun mata kuliah penganyaan. Proses ini biasanya dilakukan diawal semester sebelum proses perkuliahan
dimulai untuk mengambil mata kuliah yang akan ditempuh disemester yang baru. Dalam suatu perguruan tinggi, mahasiswa belajar dalam suatu
sistem kurikulum yang terdiri atas sekumpulan daftar mata kuliah dimana setelah menempuh mata kuliah,
mahasiswa akan mendapatkan nilai. Nilai mata kuliah yang akan diterima didasarkan atas kemampuan
mahasiswa.

Permasalahan yang biasanya dihadapi oleh mahasiswa saat akan melakukan pengambilan mata kuliah adalah kebingungan dalam
memilih mata kuliah yang sesuai minat dan kemampuan mereka. Banyak mahasiswa yang masih belum memiliki gambaran tentang
minat pada mata kuliah disemester selanjutnya. Tak sedikit pula mahasiswa yang telah memiliki perminatan tetapi masih
bingung dalam memilih mata kuliah yang akan ditempuh. Hal ini biasanya terjadi karena kurangnya pemahaman mengenai
keterkaitan antar mata kuliah yang telah, sedang dan akan ditempuh. Oleh karena itu, sistem rekomendasi sangat cocok untuk
dapat membantu mahasiswa dalam memutuskan mata kuliah apa yang akan mereka untuk ditempuh disemester selanjutnya.

Sistem rekomendasi adalah sistem yang bertujuan untuk memberikan rekomendasi item yang menarik bagi pengguna agar pengguna
dapat memutuskan item apa yang akan mereka pilih. Belakangan ini, sistem rekomendasi telah menjadi sangat populer dan menjadi
sebuah bagian yang penting diberbagai layanan. Salah satu teknik yang dapat digunakan untuk mengatasi permasalahan tersebut adalah
penerapan machine learning, khususnya metode klasifikasi untuk membantu dalam mengambil keputusan dengan menggunakan algoritma
\emph{Decision Tree}.

\subsection{Rumusan Masalah}
Berdasarkan uraian latar belakang diatas dapat dirumuskan permasalahan sebagai berikut:
\begin{enumerate}[noitemsep]
      \item Bagaimana sistem rekomendasi dapat memberikan rekomendasi mata kuliah pilihan dan pengayaan
            yang cocok dengan mahasiswa?
      \item Bagaimana mengetahui kelayakan sistem rekomendasi yang dibangun?
\end{enumerate}

\subsection{Batasan Masalah}
Batasan-batasan permasalahan yang dijadikan sebagai pedoman dalam pelaksanaan penelitian ini adalah sebagai berikut:
\begin{enumerate}[noitemsep]
      \item Sistem rekomendasi ini pemilihan mata kuliah
            ini hanya digunakan sebagai rekomendasi bagi
            mahasiswa, keputusan pemilihan mata kuliah
            ada kepada mahasiswa.
      \item Metode yang digunakan adalah item-based
            collaborative filtering dan content-based
            filtering dengan algoritma Decision Tree.
      \item Data latih yang digunakan adalah data registrasi mata kuliah mahasiswa ITS.
\end{enumerate}

\subsection{Tujuan Penelitian}
Tujuan penelitian ini dilakukan adalah:
\begin{enumerate}[noitemsep]
      \item Untuk merancang dan membuat sebuah sistem sehingga dapat memberikan rekomendasi mata kuliah pilihan
            maupun mata kuliah pengayaan yang ada yang mungkin cocok dengan mahasiswa.
      \item Untuk
            mengetahui kelayakan sistem rekomendasi yang dibangun.
\end{enumerate}

\subsection{Manfaat}
Hasil penelitian ini yang diharapkan agar dapat memberikan manfaat sebagai berikut:
\begin{enumerate}[noitemsep]
      \item Memberikan kemudahan bagi mahasiswa dalam melakukan pemilihan matakuliah yang tepat.
      \item Memberikan manfaat bagi pengembangan ilmu pengetahuan dan teknologi serta wawasan ilmu bagi
            penelitian dan pengembangan selanjutnya.
\end{enumerate}