\chapter{METODOLOGI}

\section{Metode yang digunakan}

Penelitian ini dilaksanakan sesuai dengan blok diagram pada Gambar 3.1. Blok diagram tersebut
merupakan metodologi penelitian yang disusun sesuai dengan langkah-langkah yang dilakukan dalam penelitian ini.

\begin{figure} [ht] \centering
  \includegraphics[width=160mm]{gambar/diagram-blok.png}
  \caption{Metodologi Penelitian}
\end{figure}

\section{Bahan dan peralatan yang digunakan}
\begin{enumerate}[noitemsep]
  \item Komputer/Laptop
  \item Python
  \item Jupyter Notebook
  \item Tensorflow
  \item Docker
  \item Flask
\end{enumerate}

\section{Urutan pelaksanaan penelitian}

\begin{adjustwidth}{1em}{0pt}
  \addtocontents{toc}{\protect\setcounter{tocdepth}{1}}

  \subsection{Explorasi Dataset}
  Melakukan analisa pada data untuk mendapatkan gambaran awal pada data. Analisa yang dilakukan yaitu
  memeriksa informasi pada data (tipe data pada tiap data), memeriksa \emph{missing value} {(data yang hilang pada baris data)},
  memeriksa duplikasi pada data.

  \subsection{Pre-pemrosesan Data}
  Pada proses ini dataset yang telah dilakukan analisa akan dilakukan pra-pemrosesan sebelum dataset dapat digunakan
  untuk melakukan pelatihan pada model. Pada sistem rekomendasi pra-pemrosesan data yang biasa dilakukan seperti menghapus data
  yang tidak diperlukan untuk proses pelatihan, menghapus \emph{missing value} jika ada, melakukan normalisasi pada data (biasanya mentransformasikan data kedalam range yang sama),
  melakukan pembagian data pelatihan dan data percobaan.

  \subsection{Pembuatan Model}
  Pembuatan model menggunakan dua pendekatan yaitu \emph{content-based filtering} dan \emph{collaborative filtering} dengan menggunakan metode \emph{Deep Learning}. Teknik
  \emph{content-based filtering} akan merekomendasikan mata kuliah yang benar benar mirip dengan mata kuliah yang pernah dipilih oleh mahasiswa sebelumnya. Sedangkan, \emph{collaborative filtering}
  memerlukan parameter pembanding yang akan digunakan untuk dipelajari oleh model nantinya. Pada kasus ini parameter pembanding yang akan digunakan yaitu nilai mata kuliah mahasiswa.

  \subsection{Melatih Model}
  Melakukan pelatihan pada dataset menggunakan yang telah dibuat sebelumnya untuk menemukan pola pada data dan dapat memberikan rekomendasi yang baik kepada mahasiswa.

  \subsection{Evaluasi Model}
  Evaluasi Model dilakukan dengan melakukan rekomendasi pada data percobaan. Jika dirasa rekomendasi belum cukup baik. Maka perlu melakukan
  \emph{tuning parameter} pada model seperti unit layer, jumlah layer pada model \emph{Deep Learning}, jumlah iterasi pelatihan dan beberapa parameter lain hingga model dapat memberikan rekomendasi dengan cukup baik.

  \subsection{Deploy Model}
  Pada tahap ini model dari sistem rekomendasi yang telah dibuat sebelumnya ada dideploy atau dirilis agar dapat digunakan untuk memberikan rekomendasi melalui sebuah \emph{software}.
  \emph{Software} yang digunakan untuk menampilkan rekomendasi dari mata kuliah yang diberikan oleh model tersebut berupa website sederhana yang bisa menampilkan daftar mata kuliah
  yang direkomendasikan sesuai dengan ID/NRP dari mahasiswa.

  \begin{figure} [ht] \centering
    \includegraphics[width=150mm]{gambar/mockup.png}
    \caption{Mock Up Tampilan Website Sederhana}
  \end{figure}
\end{adjustwidth}



