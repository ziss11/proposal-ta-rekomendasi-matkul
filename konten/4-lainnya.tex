\section{HASIL YANG DIHARAPKAN}

\subsection{Hasil yang Diharapkan dari Penelitian}
Dari penelitian yang akan dilakukan, diharapkan dapat menghasilkan sebuah sistem yang mampu digunakan untuk
memberikan sebuah rekomendasi mata kuliah, sebagai alat bantu untuk mahasiswa dalam membuat keputusan. Dari penelitian ini juga
diharapkan pengembangan sistem rekomendasi ini dapat mempermudah mahasiswa menemukan mata kuliah yang sesuai minat dan kemampuan mereka,
khususnya mata kuliah pilihan dan mata kuliah pengayaan, sehingga dapat meminimalkan resiko mendapatkan nilai yang kurang memuaskan saat
menempuh mata kuliah yang dipilih.

\subsection{Hasil Pendahuluan}



% \section{RENCANA KERJA}

% % Ubah tabel berikut sesuai dengan isi dari rencana kerja
% \newcommand{\w}{}
% \newcommand{\G}{\cellcolor{gray}}
% \begin{table}[h!]
%   \begin{tabular}{|p{3.5cm}|c|c|c|c|c|c|c|c|c|c|c|c|c|c|c|c|}

%     \hline
%     \multirow{2}{*}{Kegiatan} & \multicolumn{16}{|c|}{Minggu} \\
%     \cline{2-17} &
%     1 & 2 & 3 & 4 & 5 & 6 & 7 & 8 & 9 & 10 & 11 & 12 & 13 & 14 & 15 & 16 \\
%     \hline

%     % Gunakan \G untuk mengisi sel dan \w untuk mengosongkan sel
%     Pengambilan data &
%     \G & \G & \G & \G & \w & \w & \w & \w & \w & \w & \w & \w & \w & \w & \w & \w \\
%     \hline

%     Pengolahan data &
%     \w & \w & \w & \w & \G & \G & \G & \G & \w & \w & \w & \w & \w & \w & \w & \w \\
%     \hline

%     Analisa data &
%     \w & \w & \w & \w & \w & \w & \w & \w & \G & \G & \G & \G & \w & \w & \w & \w \\
%     \hline

%     Evaluasi penelitian &
%     \w & \w & \w & \w & \w & \w & \w & \w & \w & \w & \w & \w & \G & \G & \G & \G \\
%     \hline

%   \end{tabular}
% \end{table}
