{\setstretch{1.15}
\begin{center}
    \uppercase{\textbf{\large Elective and Enrichment Course Recommender System using Deep Learning}}
\end{center}

\vspace*{6 mm}
\begin{adjustwidth}{}{}
    \begin{tabular}{lcp{1\linewidth}}
        \noindent\textbf{Student Name / NRP} & : & \textbf{Abdul Azis / 0721 19 4000 7004}                \\
        \noindent\textbf{Departement}        & : & \textbf{Computer Engineering FTEIC - ITS}              \\
        \noindent\textbf{Advisor}            & : & \textbf{1. Dr. Supeno Mardi Susiki Nugroho, S.T, M.T.} \\
                                             & : & \textbf{2. Reza Fuad Rachmadi, S.T., M.T., Ph.D.}      \\
    \end{tabular}
\end{adjustwidth}

\vspace{6 mm}
\noindent
\textbf{Abstract}
\vspace{3 mm}

Students who are taking their fourth or subsequent semester will usually plan to take elective courses or enrichment courses.
The problem that is usually faced by students when taking courses is confusion in choosing courses that match their interests and abilities
they. Many students still do not have an idea about their interest in the subject in the next semester. Not a few students who already have
interest but still confused in choosing the courses to be taken. To be able to assist students in deciding whether to take courses, a course is needed
recommendation system that can provide a course recommendation. The research was conducted by looking for similarities between courses using the hybrid recommender system approach that combines the two
recommendation system approaches namely collaborative filtering and content-based filtering which by combining the two approaches will cover each other
the shortcomings of each approach in providing recommendations and using Deep Learning to build a model based on the hybrid recommender system approach. Thus, this research
aims to determine the feasibility of the recommendation system built with the approach used.

\vspace{6 mm}
\noindent
\textbf{Keyword: }Recommendation System, Hybrid Recommender System, Deep Learning
}



