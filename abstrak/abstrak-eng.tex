{\setstretch{1.15}
\begin{center}
    \uppercase{\textbf{\large Elective and Enrichment Course Recommender System using Deep Learning}}
\end{center}

\vspace*{6 mm}
\begin{adjustwidth}{}{}
    \begin{tabular}{lcp{1\linewidth}}
        \noindent\textbf{Student Name / NRP} & : & \textbf{Abdul Azis / 0721 19 4000 7004}                \\
        \noindent\textbf{Departement}        & : & \textbf{Computer Engineering FTEIC - ITS}              \\
        \noindent\textbf{Advisor}            & : & \textbf{1. Dr. Supeno Mardi Susiki Nugroho, S.T, M.T.} \\
                                             & : & \textbf{2. Reza Fuad Rachmadi, S.T., M.T., Ph.D.}      \\
    \end{tabular}
\end{adjustwidth}

\vspace{6 mm}
\noindent
\textbf{Abstract}
\vspace{3 mm}

Students who are taking their fourth or subsequent semester will usually plan to take elective courses or enrichment courses.
The problem that is usually faced by students when taking courses is confusion in choosing courses that match their interests and abilities
they. Many students still do not have an idea about their interest in the subject in the next semester. Not a few students who already have
interest but still confused in choosing the courses to be taken. Therefore, the recommendation system is perfect to be able to help
students in deciding what courses they will take in the next semester. This study aims to design and manufacture
a system so that it can provide recommendations for elective courses as well as enrichment courses that might be suitable for students, and for
determine the feasibility of the recommendation system built with the approaches used. The method used is hybrid recommender system with
using collaborative filtering and content-based filtering approaches. Collaborative filtering works by creating a database containing preferences items
from users. Collaborative filtering is able to provide recommendations that are more than just item similarities, but can provide recommendations by studying user tastes and
looking for users who have more or less the same tastes. The recommendation system with the content-based filtering method performs the learning process to recommend
items similar to the previous item that the user liked or voted for. Item similarity is calculated based on the features of the items being compared.
Therefore, this method does not depend on the situation whether the item is a new item (which has never been selected by any user) or not a new item. Content that
being a learning object in this method is generally feature information about available items, but in its development, user profiles can also be content that is learned
the resemblance. This system will be built using Deep Learning which can be more efficient and more targeted, because it can carry out complex tasks.

\vspace{6 mm}
\noindent
\textbf{Keyword: }Recommendation System, Hybrid Recommender System, Deep Learning
}



