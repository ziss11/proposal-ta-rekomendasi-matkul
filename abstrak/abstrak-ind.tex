{\setstretch{1.15}
\begin{center}
    \uppercase{\textbf{\large Sistem rekomendasi Mata Kuliah Pilihan dan Pengayaan berbasis Deep Learning}}
\end{center}

\vspace*{6 mm}
\begin{adjustwidth}{}{}
    \begin{tabular}{lcp{1\linewidth}}
        \noindent\textbf{Nama Mahasiswa / NRP} & : & \textbf{Abdul Azis / 0721 19 4000 7004}                \\
        \noindent\textbf{Departemen}           & : & \textbf{Teknik Komputer FTEIC - ITS}                   \\
        \noindent\textbf{Dosen Pembimbing}     & : & \textbf{1. Dr. Supeno Mardi Susiki Nugroho, S.T, M.T.} \\
                                               & : & \textbf{2. Reza Fuad Rachmadi, S.T., M.T., Ph.D.}      \\
    \end{tabular}
\end{adjustwidth}

\vspace{6 mm}
\noindent
\textbf{Abstrak}
\vspace{3 mm}

Mahasiswa yang sedang menempuh semester empat atau selanjutnya biasanya akan merencanakan untuk mengambil mata kuliah pilihan maupun mata kuliah penganyaan.
Permasalahan yang biasanya dihadapi oleh mahasiswa saat akan melakukan pengambilan mata kuliah adalah kebingungan dalam memilih mata kuliah yang sesuai minat dan kemampuan
mereka. Banyak mahasiswa yang masih belum memiliki gambaran tentang minat pada mata kuliah disemester selanjutnya. Tak sedikit pula mahasiswa yang telah memiliki
perminatan tetapi masih bingung dalam memilih mata kuliah yang akan ditempuh. Untuk dapat membantu mahasiswa dalam memutuskan pengambilan mata kuliah diperlukan sebuah
sistem rekomendasi yang dapat memberikan sebuah rekomendasi mata kuliah. Penelitian dilakukan dengan mencari kemiripan antar mata kuliah dengan menggunakan pendekatan \emph{hybrid recommender system} yang menggabungkan dua
pendekatan sistem rekomendasi yaitu \emph{collaborative filtering} dan \emph{content-based filtering} yang mana dengan menggabungkan dua pendekatan tersebut akan saling menutupi
kekurangan dari masing-masing pendekatan dalam memberikan rekomendasi dan menggunakan \emph{Deep Learning} untuk membangun model berbasis pendekatan \emph{hybrid recommender system}. Sehingga, penelitian ini
bertujuan untuk mengetahui kelayakan sistem rekomendasi yang dibangun dengan pendekatan yang digukanakn.

\vspace{6 mm}
\noindent
\textbf{Kata Kunci: }Recommendation System, \emph{Hybrid Recommender System}, \emph{Deep Learning}
}